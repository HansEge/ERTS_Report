%!TEX root = ../Main.tex

\chapter*{Resumé}
Denne rapport dokumenterer bachelorprojektet på Ingeniørhøjskolen Aarhus Universitet med titlen "'Safe Water in Buildings"'. Projektet er udført af to elektronik-ingeniør (E) og en informations- og kommunikationsteknologi-ingeniør (IKT) studerende, i samarbejde med firmaet Grundfos, i perioden 1. februar 2019 - 29. maj 2019 under vejledning af Carl Jakobsen.

Rammerne for projektet er baseret på en problemstilling, udarbejdet af projektgruppen i samarbejde med Grundfos.

Projektet omhandler design og implementering af en prototype på et intelligent hjemme vandrensnings system, til rensning af vand i mindre bygninger (1 -2 familier). Herudover skal det være muligt for en bruger at få opbygget sit eget systemt på præcis den måde de ønsker, så deres personlige behov er mødt, derfor skal systemet designes til at tillade denne dynamiske opbygning.

Realiseringen af projektet sker gennem henholdsvis et "'Windows Presentation Forms"' (WPF) projekt, der fungere som grafisk grænseflade (GUI) for systemet, samt et \CS-projekt der fungerer som controller for systemet. Gennem systemets GUI, kan brugeren tilgå alle funktionaliteter for systemet, og aflæse data fra dette. Derudover er en "'Programmable Logic Controller"' (PLC) indraget som har fungeret som grænseflade til sensorer tilknyttet systemet.

Udviklings- og designprocessen tog udgangspunkt i en modificeret versions af projektstyrings værktøjet SCRUM. Gruppen arbejde iterativt, i sprints af omkring en uges varighed, herved sikredes en overskuelig of effektiv arbejdsprocess.

Projektet udmundede i et system, der opfyldte gruppens forventninger.